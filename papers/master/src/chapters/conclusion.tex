\chapter{Conclusion}
\label{chapter:conclusion}

In this paper, a viable theoretical framework for identifying historic events in texts has been described. Notions like the historical relevance of a word and the historically relevant document of a year are a key component of the framework. While not being clearly defined, they manage to create an abstraction that can be used to explore and compare various possibilities of identifying historic events.

Furthermore, several possible definitions for the above notions have been proposed. Upon implementing these, a study of the results has shown that while most of the important historic events are represented in the discovered topics in at least one of the models, there is not a single one that can claim supremacy over the others.

One of the most important directions for future work is exploring alternatives for some of the steps in the workflow. For example, one can wonder if topic models are really the key of understanding the historically relevant documents, or if a better model can be used. Also, the notion of historical relevance has a lot of potential for improvement, both in finding radically new definitions and in finding ways to combine the existing ones. The latter idea seems to be the most promising, since the sets of historical events identified by each algorithm have a significant number of events that are not common to all of them.

Secondly, one can try to refine the current model. A possible area for improvement is the processing of topics that resulted from applying Latent Dirichlet Allocation, which could use a way of splitting the topics into sub-topics that could be associated with historic events.

Last, but not least, there is a need for a way of comparing the results produced when using different algorithms. Currently, the main words forming a topic vary wildly across methods, so it is quite difficult to perform such an analysis, but if the improvement concerning sub-topics succeeds, it can definitely be used as a starting point for this one.
