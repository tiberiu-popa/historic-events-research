% vim: set tw=78 sts=2 sw=2 ts=8 aw et ai:

In \newcite{Michel14012011}, the authors have laid the foundation of culturomics, i.e. the quantitative analysis of culture. Thus, using a corpus of over 5 million books, they have managed to computationally investigate cultural trends.

The Google Books Ngram Corpus consists of word n-gram statistics, the most important being the number of occurrences for each n-gram in books written in a given year. Using this data, one can compute the distribution of n-grams that appeared in books written in each year. The most interesting applications of culturomics lie in social sciences such as history. For example, one can deduce from the plot associated with influenza when the most devastating flu pandemics have occurred: they correspond to the peaks in the time series.

This kind of analyses can also be applied to more recent texts, such as news archives. In \newcite{leetaru11culturomics}, Leetaru has used data from the Summary of World Broadcasts to analyze events such as the Arab Spring or the ethnic conflicts that occurred in Serbia in the 1990s.

\newcite{Hall:2008:SHI:1613715.1613763} have studied a related question, focusing on the history of ideas in the field of Computational Linguistics. By analyzing the ACL Anthology using topic models, they revealed historical trends such as the rise of probabilistic models over the last 25 years. In another influential study, \newcite{Wijaya:2011:USC:2064448.2064475} have used the Google Books Ngram Corpus to analyze the semantic evolution of words over centuries. The core idea is to consider words that co-occur with a given word and then to apply topic modeling to them, the end result being topics which correspond to historical meanings of a word. The particular model used is Topics over Time (TOT), first introduced by \newcite{Wang:2006:TOT:1150402.1150450}.
