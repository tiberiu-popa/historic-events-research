% vim: set tw=78 sts=2 sw=2 ts=8 aw et ai:

In this paper, a viable theoretical framework for identifying historic events in texts has been described, its key component being the notion of historical relevance of a word. Although not clearly defined, it manages to create an abstraction that can be used to explore and compare various possibilities of identifying historic events.

Furthermore, several possible definitions for the above notions have been proposed. Upon implementing these, a study of the results has shown that while most of the important historic events are represented in the discovered topics in at least one of the models, there is not a single one that can claim supremacy over the others.

One of the most important directions for future work is exploring alternatives for some of the steps in the workflow. For example, one can wonder if topic models are really the key of understanding the historically relevant documents, or if a better model can be used. Also, the notion of historical relevance has a lot of potential for improvement, both in finding new definitions and in finding ways to combine the existing ones.

Furthermore, there is a need for a way of automatically comparing the results produced when using different algorithms. Currently, the main words forming a topic vary wildly across methods, so it is quite difficult to perform such an analysis even by hand, but this issue will no doubt be a target of future research.
